\begin{exercise}{10+15}
Since there are languages that can not be lexically analysed using the longest-matching principle, we derive an alternative automata model for the lexical analysis. For that, we use nondeterminstic Mealy-Automata (NMA). An NMA is a 6 tuple $\mathcal{M}=(Q, \Omega, \Sigma, \Delta, q_0, F)$ where:
\begin{itemize}
    \item $Q$ is a finite set of all states,
    \item $\Omega$ is a set of input symbols,
    \item $\Sigma$ is a set of output symbols (with $\epsilon \not\in \Sigma$),
    \item $\Delta \subseteq Q \times \Omega \times (\Sigma \cup \{\epsilon\}) \times Q$ is a transition relation,
    \item $q_0 \in Q$ is the initial state, and
    \item $F \subseteq Q$ is the set of final states.
\end{itemize} 
%
NMAs extend nondeterministic finite automata by \emph{outputs}:
Given an (input) word $w \in \Omega^*$, an NMA $\mathcal{M}$ determines a language of (output) words $O \subseteq \Sigma^*$ as follows:
An accepting run of $\mathcal{M}$ on a word $w\in \Omega^*$  starts in the initial state $q_0$ and ends in some final state $q \in F$. The set of accepting runs on $w$ is 
%
\[ 
    \mathcal{M}(w) = \left\{ (a_1, z_1, q_1) \dots (a_n, z_n, q_n) \in (\Omega \times (\Sigma \cup \{\epsilon\}) \times Q)^* \middle| 
    \begin{array}{c}
        \forall 0 < i \leq n ~ (q_{i-1}, a_i, z_i, q_i) \in \Delta,\\
        a_1 \dots a_n = w, q_n \in F 
    \end{array}\right\}~.
\]
%
Notice that a transition may produce no symbol, i.e., an $\epsilon$, and that the first transition always starts in $q_0$. The language determined by $\mathcal{M}$ on input word $w \in \Omega^*$ is then defined as
%
\[ L_{\mathcal{M}}(w) = \{ z_1 \dots z_n \in \Sigma^* \mid \exists (a_1, z_1, q_1) \dots (a_n, z_n, q_n) \in \mathcal{M}(w) \}~. 
\]
%
%Eventuell brauchen wir das gar nicht:
%Lastly we define the language of a NMA $\mathcal{M}$ as all possible output words, i.e. the language of $\mathcal{M}$ given all input words:
%\[ L(A) = \bigcup_{w \in \Sigma^*} L_A(w). \]

\begin{enumerate}[a)]
\item Consider the following NMA $\mathcal{M}=(Q, \Omega, \Sigma, \Delta, q_0, F)$:
    \begin{itemize}
        \item $Q=\{q_0, q_1, q_2, q_3\}$,
        \item $\Omega=\{a,b,c\}$,
        \item $\Sigma=\{\alpha, \beta, \gamma\}$,
        \item $\Delta=\{(q_0,b,\epsilon,q_0), (q_0,c,\epsilon,q_0), (q_0,a,\epsilon,q_1), (q_1,a,\epsilon,q_1), (q_1,b,\epsilon,q_2),$\\
                       $(q_2,a,\epsilon,q_2), (q_2,b,\epsilon,q_3), (q_2,a,\alpha,q_0), (q_2,b,\beta,q_0), (q_3,c,\gamma,q_2)\}$ and
        \item $F=\{q_0\}$.
    \end{itemize}
     %
    We can depict $\mathcal{M}$ graphically as follows:
    \begin{center}
    \begin{tikzpicture}
    \tikzset{
        ->, % makes the edges directed
        >=latex, %other arrowhead
        node distance=3cm, % specifies the minimum distance between two nodes. Change if necessary.
        every state/.style={thick, fill=gray!10}, % sets the properties for each ’state’ node
        initial text=$ $, % sets the text that appears on the start arrow
    }
        \node[state, initial, accepting] (0) {$q_0$};
        \node[state, right of=0] (1) {$q_1$};
        \node[state, right of=1] (2) {$q_2$};
        \node[state, right of=2] (3) {$q_3$};
        
        \draw (0) edge[loop above] node{$b/\epsilon$, $c/\epsilon$} (0)
              (0) edge[above] node{$a/\epsilon$} (1)
              (1) edge[loop above] node{$a/\epsilon$} (1)
              (1) edge[above] node{$b/\epsilon$} (2)
              (2) edge[loop above] node{$a/\epsilon$} (2)
              (2) edge[above, bend left] node{$b/\epsilon$} (3)
              (3) edge[below, bend left] node{$c/\gamma$} (2)
              (2) edge[below, bend left] node{$a/\alpha$, $b/\beta$} (0);
    \end{tikzpicture}
    \end{center}
    Give $L_{\mathcal{M}}(abaaabcab)$.
    %
    %
    \item  In this task, we employ NMAs to solve a modification of the extended matching problem (cf.\ Lecture~3, Slide~8): Let $n \geq 1$ and $\alpha_1,\ldots,\alpha_n \in RE_\Omega$ with $\epsilon \not\in L(\alpha_i) \ne \emptyset$ for every $i \in [n]$. Let $\Sigma = \{T_1,\ldots,T_n\}$ be an alphabet of corresponding tokens. Construct an NMA $\mathcal{M}$ such that for every $w \in \Omega^+$,
    %
    \begin{equation*}
         L_{\mathcal{M}}(w) = \left\{  T_{i_1},\ldots,T_{i_k} \in \Sigma^+ ~\middle|~ \begin{aligned}(T_{i_1},\ldots,T_{i_k})~\text{is a corresponding analysis of some} \\
         %
         \text{decomposition}~(w_1,\ldots,w_k)~\text{of}~w~\text{w.r.t.}~\alpha_1,\ldots,\alpha_n \end{aligned} \right\}~.
    \end{equation*}
    %
    Justify the correctness of your construction. \\ \\
    %
    \noindent
    \emph{Hint: In this task, we do not apply the first longest match principle. Rather, we consider \emph{all} possible analyses of a word $w \in \Omega^+$.}
\end{enumerate}
\end{exercise}

\begin{solution}
\begin{enumerate}[a)]
\item $L_{\mathcal{M}}(abaaabcab)=\{\gamma \beta, \gamma \alpha \}$
%
%
\item We construct the NMA $\mathcal{M} = (Q, \Omega, \Sigma, \Delta, q_0, F)$ as follows:
%
\begin{enumerate}
	\item For every $\alpha_i$, construct a DFA $\mathfrak{A}_i = (Q_i, \Omega, \delta_i, q_{0_i} F_i)$. 
	%
	\item Construct the product automaton $\mathfrak{A} = (Q', \Omega, \delta, q_0', F')$ w.r.t.\ $\mathfrak{A}_1, \ldots, \mathfrak{A}_n$.
	%
	\item Now construct $\mathcal{M}$ by
	%
	\begin{itemize}
		\item $Q = Q' \cup \{q_{match}\}$
		%
		\item ($\Omega$ is the alphabet of the $\alpha_i$ and $\Omega =\{T_1, \ldots, T_n\}$ is the alphabet of tokens.)
		%
		\item the transition relation $\Delta$ is the smallest relation satisfying
		%
		\begin{align*} 
		  ((q_1, \ldots, q_n), a, \epsilon, (q_1',\ldots,q_n')) \in \Delta \qquad \text{if} \qquad \delta((q_1, \ldots, q_n), a) = (q_1',\ldots,q_n')
		  \tag{simulate $\mathfrak{A}$, thus continue searching for a matching}
		\end{align*}
		%
		and
		%
		\begin{align*} 
		((q_1, \ldots, q_n), a, T_i, q_{match}) \in \Delta \qquad \text{if} \qquad \delta((q_1, \ldots, q_n, a) = (q_1',\ldots,q_n')~\text{and}~q_i' \in F_i~,
		\tag{on reaching a final state of some $\mathfrak{A}_i$, produce corresponding token $T_i$}
		\end{align*}
		%
		and
		%
		\begin{align*} 
		(q_{match}, a, o, q') \in \Delta \qquad \text{if} \qquad (q_0, a, o, q') \in \Delta
		\tag{simulate restarting $\mathcal{A}$ by copying transitions of the inital state to the matching state}
		\end{align*}
		%
		\item $q_0 = q_0'$ 
		%
		\item $F = \{q_{match}\}$.
	\end{itemize}
    %
    %
\end{enumerate}
Our idea here is to simulate the product automaton over all tokens. As soon as a final state is reached, then can either continue searching for a match, or go back to a matching state that simulated the initial state and output a matched token. We need the nonderminism to allow the automaton to choose between continuing searching for a matching or matching the infix. We furthermore need to choose a matching token nondeterministically, since we do not have any priorities any more (in contrast to a first matching where we take the token with the highest priority).
\end{enumerate}
\end{solution}

