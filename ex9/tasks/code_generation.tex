\begin{exercise}{30}
Consider the following EPL program:
\begin{lstlisting}[numbers=none]
in/out x;
var y;
proc P;
	x := 3 * x
end;
proc Q;
	x := x - y
end;
if x <= 100 or y == x then
	P()
else
    y := 5;
	Q()
end
end.
\end{lstlisting}


Translate the EPL program to abstract machine (AM) code.

\end{exercise}

\begin{solution}
    The AM code is as follows:
    \begin{lstlisting}[numbers=none]
        1: CALL(13,0,1); // Start: call Main()
        2: JMP(0);       // Terminated
        3: LIT(3);       // Begin P()
        4: LOAD(2,1)
        5: MULT;
        6: STORE(2,1);   // x := 3 * x
        7: RET;          // End P()
        8: LOAD(2,1);    // Begin Q()
        9: LOAD(1,1);
       10: SUB;
       11: STORE(2,1);   // x := x - y
       12: RET;          // End Q()
       13: LOAD(1,1);    // Begin Main()
       14: LIT(100);
       15: LEQ;          // x <= 100
       16: LOAD(0,1);
       17: LOAD(1,1);
       18: EQ;           // y == x
       19: OR;           // x <= 100 or y == x
       20: JFALSE(23);
       21: CALL(3,0,0);  // P()
       22: JMP(26);
       23: LIT(5);
       24: STORE(0,1);   // y := 5
       25: CALL(8,0,0);  // Q()
       26: RET;          // End Main()
    \end{lstlisting}
\end{solution}
