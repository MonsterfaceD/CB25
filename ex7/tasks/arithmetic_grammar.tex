\begin{exercise}{30}

In this exercise we apply attribute grammars to evaluate simple arithmetic expressions.
\begin{enumerate}[(a)]
  \item Write an \emph{unambiguous} context-free grammar for arithmetic expressions that contains arbitrary numerical values (which you may represent by a single terminal symbol \texttt{num}), addition (\texttt{+}), multiplication (\texttt{*}), and parentheses (\texttt{(}, \texttt{)}).\\
    \emph{Hint:} You can already incorporate the usual operator precedence (\texttt{(}, \texttt{)} > \texttt{*} > \texttt{+}) into the grammar.
  \item Define an attribute $\textit{value}$ to compute the value of an expression.
    That is, the value of a given expression should correspond to the value of attribute $\textit{value}$ at the root of the corresponding derivation tree.
    Make sure to follow the operator precedence as given in (a).
  \item Evaluate \texttt{(1+3*2)*6}.
    To this end, first construct the corresponding derivation tree, then set up the equation system, and solve it.
\end{enumerate}
\end{exercise}

\begin{solution}
\begin{enumerate}[(a)]
  \item The unambiguous CFG $G = (N, \Sigma, P, S)$ for the arithmetic expression is given as follows:
    \begin{itemize}
        \item $N := \set{S, M, N}$
        \item $\Sigma := \set{\texttt{+}, \texttt{*}, \texttt{(}, \texttt{)}, \texttt{num}}$
        \item $P:$
        \begin{align*}
            S & \to S \texttt{ + } M \mid M \\
            M & \to M \texttt{ * } N \mid N \\
            N & \to \texttt{(} S \texttt{)} \mid \texttt{num}
        \end{align*}
    \end{itemize}

  \item The attribute grammar for the CFG $G$ of (a) has the synthesized attribute $\set{Val}$ and is given as follows:
    \[
    \begin{array}{lcll}
        S & \to & S \texttt{ + } M        & Val.0 = Val.1 + Val.3 \\
        S & \to & M                       & Val.0 = Val.1 \\
        M & \to & M \texttt{ * } N        & Val.0 = Val.1 \cdot Val.3 \\
        M & \to & N                       & Val.0 = Val.1 \\
        N & \to & \texttt{(} S \texttt{)} & Val.0 = Val.2 \\
        N & \to & \texttt{num}            & Val.0 = Val.1
    \end{array}
    \]

  \item The annotated parse tree for \texttt{(1+3*2)*6} is as follows:
    \begin{center}
    \begin{tikzpicture}
        [node distance = 1.5cm, auto]
        [->,>=stealth',shorten >=1pt,auto,node distance=1cm, semithick]
        %    \tikzstyle{every state}=[circle, radius = 2pt]
        \tikzstyle{state}=[circle,draw=black,fill=gray,minimum size=5pt,inner sep=0pt]

        \node[] (0) {S: $v_0$};

        \node[] (1) [below of=0] {M: $v_1$};

        \node[] (3) [below of=1] {\texttt{*}};
        \node[] (2) [left=2.5cm of 3] {M: $v_2$};
        \node[] (4) [right=2.5cm of 3] {N: $v_{12}$};

        \node[] (5) [below of=2] {N: $v_3$};
        \node[] (6) [below of=4] {\texttt{num}: 6};

        \node[] (8) [below of=5] {S: $v_4$};
        \node[] (7) [left=2cm of 8] {\texttt{(}};
        \node[] (9) [right=2cm of 8] {\texttt{)}};

        \node[] (10) [below of=8] {\texttt{+}};
        \node[] (11) [left=1.5cm of 10] {S: $v_5$};
        \node[] (12) [right=1.5cm of 10] {M: $v_8$};

        \node[] (13) [below of=11] {M: $v_6$};
        \node[] (14) [below of=12] {\texttt{*}};
        \node[] (15) [left of=14] {M: $v_9$};
        \node[] (16) [right of=14] {N: $v_{11}$};

        \node[] (17) [below of=13] {N: $v_7$};
        \node[] (18) [below of=16] {\texttt{num}: 2};
        \node[] (19) [below of=15] {N: $v_{10}$};

        \node[] (20) [below of=17] {\texttt{num}: 1};
        \node[] (21) [below of=19] {\texttt{num}: 3};

        \path
        (0)  edge (1)
        (1)  edge (2)
             edge (3)
             edge (4)
        (2)  edge (5)
        (4)  edge (6)
        (5)  edge (7)
             edge (8)
             edge (9)
        (8)  edge (10)
             edge (11)
             edge (12)
        (11) edge (13)
        (12) edge (14)
             edge (15)
             edge (16)
        (13) edge (17)
        (15) edge (19)
        (16) edge (18)
        (17) edge (20)
        (19) edge (21)
        ;
    \end{tikzpicture}
    \end{center}

The set of variables is $\set{v_0,\hdots,v_{12}}$. The equation system is defined as follows, where bold expressions on the right-hand side result from solving the system.
\[
\begin{array}{lcll}
    Val.v_0    & = & Val.v_1              & = \textbf{42} \\
    Val.v_1    & = & Val.v_2 \cdot v_{12} & = \textbf{42} \\
    Val.v_2    & = & Val.v_3              & = \textbf{7} \\
    Val.v_3    & = & Val.v_4              & = \textbf{7} \\
    Val.v_4    & = & Val.v_5 + v_8        & = \textbf{7} \\
    Val.v_5    & = & Val.v_6              & = \textbf{1} \\
    Val.v_6    & = & Val.v_7              & = \textbf{1} \\
    Val.v_7    & = & 1                    & = \textbf{1} \\
    Val.v_8    & = & Val.v_9 \cdot v_{11} & = \textbf{6} \\
    Val.v_9    & = & Val.v_{10}           & = \textbf{3} \\
    Val.v_{10} & = & 3                    & = \textbf{3} \\
    Val.v_{11} & = & 2                    & = \textbf{2} \\
    Val.v_{12} & = & 6                    & = \textbf{6} \\
\end{array}
\]
\end{enumerate}

\end{solution}
