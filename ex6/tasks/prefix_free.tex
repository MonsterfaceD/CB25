\begin{exercise}{30}
\begin{mydef}[Prefix Free]
A language $L \subseteq \Sigma^{*}$ is called prefix free if $L \cdot \Sigma^{+} \cap L = \emptyset$, i.e., if no proper prefix of an
element of $L$ is again in $L$.
\end{mydef}

Let $L$ be a non-empty regular language. Prove:
$$L \text{ can be generated by an } LR(0) \text{ grammar } \quad  \text{iff} \quad L \text{ is prefix free}$$
\end{exercise}

\begin{solution}
First, we consider the direction $\Rightarrow$.\\
Assume $G\in LR(0)$ and $L(G)$ is not prefix free. Thus there exists $v \in \Sigma^{*}$ and $w \in \Sigma^{+}$ such that $v, vw \in L(G)$. Then the $LR(0)$ parsing automaton for $G$ has the following accepting run on $v$:
\[ (v, I_0, \varepsilon) \vdash^{*} (\varepsilon, I_0 I, z) \vdash (\varepsilon, \varepsilon, z0) \text{ where } act(I) = accept\]
and for $vw$:
\[ (vw, I_0, \varepsilon) \vdash^{*} (w, I_0 I, z) \vdash^{+} (\varepsilon, I_0 I', zz') \vdash (\varepsilon, \varepsilon, zz'0) \text{ where } act(I) \neq accept\]
which contradicts $act(I)=accept$.\par

We consider the direction $\Leftarrow$.\\
Let $L$ be regular and prefix free.
Because $L$ is regular it is recognisable by a ``partial'' DFA $\mathfrak{A}$ with only productive states (i.e., the transition function is deterministic but not necessarily total).\\
In particular, since $L$ is prefix-free, no final state has outgoing transitions (as these could only lead to non-productive states).
The transformation of $\mathfrak{A}$ into an equivalent CFG (similar to Exercise 4.5)
thus yields a grammar $G$ where, for all states $q$, we have either productions of the form
\[
\begin{array}{lll}
    (i)  & q \to a_i q_j& \text{ or of the form} \\
    (ii) & q \to \varepsilon  & ~.\\
\end{array}
\]
Rules (i) and (ii) are mutually exclusive, because $L$ is prefix free.

For this grammar $G$, where we add start-separation, the construction of $LR(0)$ sets yields:
\[
\begin{array}{lcll}
LR(0)(\varepsilon): & (i)   & {[S \to \cdot q_0], [q_0 \to \cdot a_1 q_1], \hdots, [q_0 \to \cdot a_k q_k]} & (\text{shift}) \\
LR(0)(\varepsilon): & (ii)  & {[S \to \cdot q_0], [q_0 \to \varepsilon \cdot]} & (\text{reduce}) \\
LR(0)(q_0):         &       & {[S \to q_0 \cdot ]} & (\text{accept}) \\
LR(0)(w a):         & (i)   & {[q \to a \cdot q'], [q' \to \cdot a_1' q_1'], \hdots, [q' \to \cdot a_l' q_l'] } & (\text{shift}) \\
                    & (ii)  & {[q \to a \cdot q'], [q' \to \varepsilon \cdot]} & (\text{reduce})\\
                    & (iii) & \emptyset & (\text{error})\\
LR(0)(w a q'):      & (i)   & {[q \to a q' \cdot]} & (\text{reduce})\\
                    & (ii)  & \emptyset & (\text{error})\\
\end{array}
\]
We cover all sets, since starting from $\varepsilon$ we can either progress with the non-terminal $q_0$, reduce with $\varepsilon$ or any terminal $a_i$ for which $q_0$ has an outgoing transition. For any word consisting only of terminal symbols $wa$ we prove by induction on the length of the word $|wa|=n$ that every set has any of the three forms shown above. 

For $n=1$ we compute from $LR(0)(\varepsilon)$ the set $LR(0)(a)$ by shifting the dot over $a$. Since $\mathfrak{A}$ is deterministic, only one production rule applies and thus we either get 
\begin{itemize}
    \item $[q_0 \to a \cdot q']$ and for all productions $q' \to a_i' q_i'$ the item $[q' \to \cdot a_i' q_i']$\\
    \emph{(if the transition from $q_0$ with $a$ goes to a non-final state)};
    \item $[q_0 \to a \cdot a']$ and for the (single) production $q' \to \varepsilon \cdot$ the item $[q' \to \varepsilon \cdot ]$\\
    \emph{(if the transition from $q_0$ with $a$ goes to a final state)};
    \item or $\emptyset$\\
    \emph{(if there is no transition from $q_0$ with $a$)}.
\end{itemize}

Now we assume the statement holds for a fixed but arbitrary length $n$.

For $|wa|=n+1$ we have by the induction hypothesis that $LR(0)(w)$ has one of the forms above. For form $(i)$, the same reasoning as in the induction base holds. For form $(ii)$, $LR(0)(wa)=\emptyset$ since form $(ii)$ does not allow shifting over a terminal (but only non-terminal). For form $(iii)$, $LR(0)(wa)=\emptyset$ by definition of $LR(0)$ sets.

We can furthermore derive that $LR(0)(w a q')$ has the form as depicted above and that no other $LR(0)$ set can be computed.
Since no $LR(0)$ set has a conflict (remark that $(ii)$ in $LR(0)(w a)$ is not a reduce-shift conflict since the shift is over a non-terminal), the grammar is $LR(0)$.
\end{solution}
