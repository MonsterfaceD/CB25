\documentclass[a4paper]{article}

% Packages
\usepackage{amsmath}
\usepackage{amssymb}
\usepackage{amsthm}
\usepackage{bold-extra}
\usepackage{fancyhdr}
\usepackage{geometry}
\usepackage{graphicx}
\usepackage{hyperref}
\usepackage{ifthen}
\usepackage[utf8]{inputenc}
\usepackage{multirow}
\usepackage{needspace}
\usepackage{parskip}
\usepackage{stmaryrd}
\usepackage{listings}
\usepackage[T1]{fontenc}
\usepackage{longtable}
\usepackage{comment}
\usepackage{enumerate}
\usepackage{xspace}
\usepackage{textcomp}
\usepackage{array}

\usepackage{tikz}
\usetikzlibrary{automata,positioning,shapes.geometric}
\usepackage{tikzsymbols}
\usepackage{todonotes}

\lstset{language=Java, numbers=left, showstringspaces=false, tabsize=4}
\geometry{a4paper, left=25mm,right=25mm, top=25mm, bottom=25mm}

% check for the existence of commands
\newcommand{\checkfor}[3]{%
  \ifcsname#1\endcsname%
  #2
  \else%
  #3
  \fi%
}

\checkfor{exnumber}{}{\newcommand{\exnumber}{-1}}

\newcommand{\exercisepagebreak}{\checkfor{isexercise}{\pagebreak}{}}
\newcommand{\solutionpagebreak}{\checkfor{isexercise}{}{\pagebreak}}

\setcounter{section}{\exnumber{}}

\numberwithin{equation}{section}
\numberwithin{figure}{section}
\numberwithin{table}{section}
\renewcommand{\qedsymbol}{\textsc{q.e.d.}}
\renewenvironment{proof}[1][\proofname]{{\bfseries #1: }}{\qed}
\newtheoremstyle{defstyle}{10pt}{5pt}{\addtolength{\leftskip}{2\leftmargini}\addtolength{\rightskip}{2\leftmargini}}{-1\leftmargini}{\scshape\bfseries}{:}{\newline}{#1 #2\ifthenelse {\equal {#3}{}} {}{ (\text{\textsc{#3}})}}{}
\newtheoremstyle{thmstyle}{10pt}{5pt}{\addtolength{\leftskip}{2\leftmargini}\addtolength{\rightskip}{2\leftmargini} \slshape}{-1\leftmargini}{\scshape\bfseries}{:}{\newline}{#1 #2\ifthenelse {\equal {#3}{}} {}{ (\text{\textsc{#3}})}}{}
\newtheoremstyle{exstyle}{10pt}{5pt}{\addtolength{\leftskip}{2\leftmargini}\addtolength{\rightskip}{2\leftmargini}}{-1\leftmargini}{\scshape\bfseries}{:}{\newline}{#1 #2\ifthenelse {\equal {#3}{}} {}{ (\text{\textsc{#3}})}}{}
\newtheoremstyle{algostyle}{10pt}{5pt}{\addtolength{\leftskip}{2\leftmargini}\addtolength{\rightskip}{2\leftmargini}}{-1\leftmargini}{\scshape\bfseries}{:}{\newline}{#1\ifthenelse {\equal {#3}{}} { #2}{ \text{\textsc{#3}}}}{}
\theoremstyle{defstyle}
\newtheorem{mydef}{Definition}[section]
\theoremstyle{thmstyle}
\newtheorem{mythm}{Theorem}[section]
\newtheorem{mylem}[mythm]{Lemma}
\newtheorem{myprop}[mythm]{Proposition}
\theoremstyle{exstyle}
\newtheorem{myex}{Example}[section]
\theoremstyle{algostyle}
\newtheorem{myalgo}{Algorithm}

% Define programming and solution environment and only use if enabled
\checkfor{isprog}{
  % Define exercise environment
  \newcounter{exercise}
  \newenvironment{exercise}[1]{\refstepcounter{exercise}\label{ex\theexercise}\section*{Programming Exercise \theexercise \hfill (#1 Points)}}{}
  \checkfor{isexercise}{
    % Programming exercise
    \excludecomment{solution}
    \excludecomment{onlysolution}
    \newenvironment{onlyexercise}{}{}
    \newcommand{\extitle}{Programming Exercise}
  }{
    % Programming solution
    \newenvironment{solution}{\label{sol\theexercise}\subsection*{Solution: \hrulefill}}{}
    \newenvironment{onlysolution}{}{}
    \excludecomment{onlyexercise}
    \newcommand{\extitle}{Programming Solution}
    }
}{
  % Define exercise environment
  \newcounter{exercise}
  \newenvironment{exercise}[1]{\refstepcounter{exercise}\label{ex\theexercise}\section*{Exercise \theexercise \hfill (#1 Points)}}{}
  \checkfor{isexercise}{
    % Theoretical exercise
    \excludecomment{solution}
    \excludecomment{onlysolution}
    \newenvironment{onlyexercise}{}{}
    \newcommand{\extitle}{Exercise Sheet}
  }{
    % Theoretical solution
    \newenvironment{solution}{\label{sol\theexercise}\subsection*{Solution: \hrulefill}}{}
    \newenvironment{onlysolution}{}{}
    \excludecomment{onlyexercise}
    \newcommand{\extitle}{Solution}
  }
}

% Define header
\pagestyle{fancy}
\fancyhf{} % Clear all headers
\setlength{\headsep}{25pt}
\cfoot{\thepage} % Page numbers
\lhead{ % Header-Definition
  % Logo
  \begin{tabular}[b]{l l}
      \multirow{2}{38mm}{
        \raisebox{-3.6mm}[0pt][0pt]{
          \includegraphics[height=14mm]{../i2}
        }
      }
      & Lehrstuhl f{\"u}r Informatik 2 \\
      & Software Modeling and Verification
    \end{tabular}
}
\rhead{ % Header-Definition
  % Course name
  \begin{tabular}[b]{r}
    Compiler Construction 2025\\
    \extitle{} \exnumber
  \end{tabular}
}
\AtBeginDocument{
  \vspace*{-30pt}
  apl.\ Prof.\ Dr.\ Thomas Noll\hfill Daniel Zilken, Roy Hermanns
  \vspace{5mm}
}


\newcommand{\header}[1]{
  % Header
  \begin{center}
    {\huge \textbf{Compiler Construction 2025}}\\
    \vspace*{1\baselineskip}%
    {\huge \textbf{--- \extitle{} \exnumber{} ---}}\\
    \checkfor{isexercise}{
      \vspace*{1\baselineskip}
      \checkfor{isprog}{
        %Upload in Moodle until #1 before the exercise class.
      }{
        Upload in Moodle or hand in until #1 before the exercise class.
      }
    }{}
    \vspace*{1.5\baselineskip}
    \hrule
  \end{center}
}

% Change numbering to (a) and (i)
\renewcommand{\labelenumi}{(\alph{enumi})}
\renewcommand{\labelenumii}{(\roman{enumii})}

% Custom commands
\newcommand{\TODO}[1]{\color{red}\textbf{TODO:} #1\color{black}}

% Macros
\newcommand{\set}[1]{\ensuremath{\left\{ #1 \right\}}}
\newcommand{\Nats}{\ensuremath{\mathbb{N}}}
\newcommand{\Reals}{\ensuremath{\mathbb{R}}}

\newcommand{\PTIME}{\mbox{\rm PTIME}}
\newcommand{\PSPACE}{\mbox{\rm PSPACE}}
\newcommand{\coNP}{\mbox{\rm coNP}}
\newcommand{\NP}{\mbox{\rm NP}}
\newcommand{\poly}{\mbox{\rm poly}}
\newcommand{\coPTIME}{\mbox{\rm coPTIME}}
\newcommand{\coPSPACE}{\mbox{\rm coPSPACE}}
\newcommand{\NPSPACE}{\mbox{\rm NPSPACE}}
\def\EXPTIME{\text{\rm EXPTIME}}
\def\doubleEXPTIME{\text{\rm 2EXPTIME}}

% Lecture specific commands
\renewcommand{\L}{{\cal L}}
\newcommand{\numberone}{\ensuremath{\set{1, \dots, 9}}}
\newcommand{\numberzero}{\ensuremath{\set{0, \dots, 9}}}
\newcommand{\eps}{\ensuremath{\varepsilon}}
\newcommand{\sem}[1]{\llbracket#1\rrbracket}
\newcommand{\la}{\ensuremath{\textsf{la}}}
\newcommand{\fir}{\ensuremath{\textsf{fi}}}
\newcommand{\first}{\ensuremath{\textsf{first}}}
\newcommand{\fo}{\ensuremath{\textsf{fo}}}
\newcommand{\follow}{\ensuremath{\textsf{follow}}}
\newcommand{\cyl}[1]{\ensuremath{\mathit{Cyl}(#1)}}
\newcommand{\icompiler}[0]{\texttt{i2Compiler}}
\newcommand{\while}[0]{\textit{WHILE}\xspace}



\begin{document}

\header{January 20th}

%\section*{General Remarks}
%\begin{itemize}
%  \item This programming exercise is optional and provides bonus points for the exercise class.
%\end{itemize}

\begin{exercise}{10}
  We now finish the implementation of our \icompiler{} by generating the corresponding \texttt{Jasmin} code for our parsed program.

  As stated on the \href{http://jasmin.sourceforge.net/}{Jasmin Webpage}:\\
  ``\emph{Jasmin is an assembler for the Java Virtual Machine. It takes ASCII descriptions of Java classes, written in a simple assembler-like syntax using the Java Virtual Machine instruction set. It converts them into binary Java class files, suitable for loading by a Java runtime system.}''

  Our compiler generates code for the Jasmin language from the parsed \textit{While} language. As a recap the grammar for the \textit{While} language is given as follows:
  \begin{center}
  \begin{tabular}{lll}
    start       &$\to$ & program EOF\\
    program     &$\to$ & statement program $\mid$ statement\\
    statement   &$\to$ & declaration SEM $\mid$ assignment SEM $\mid$ branch $\mid$ loop $\mid$ out SEM\\
    declaration &$\to$ & INT ID\\
    assignment  &$\to$ & ID ASSIGN expr $\mid$ ID ASSIGN READ LBRAC RBRAC\\
    out         &$\to$ & WRITE LBRAC expr RBRAC $\mid$ WRITE LBRAC STRING RBRAC\\
    branch      &$\to$ & IF LBRAC guard RBRAC LCBRAC program RCBRAC $\mid$\\
                &      & IF LBRAC guard RBRAC LCBRAC program RCBRAC\\
                &      & ELSE LCBRAC program RCBRAC\\
    loop        &$\to$ & WHILE LBRAC guard RBRAC LCBRAC program RCBRAC\\
    expr        &$\to$ & NUMBER $\mid$ ID $\mid$ subexpr\\
    subexpr     &$\to$ & expr PLUS expr $\mid$ expr MINUS expr $\mid$ expr TIMES expr $\mid$ expr DIV expr\\
                &      & $\mid$ expr MOD expr\\
    guard       &$\to$ & relation $\mid$ subguard $\mid$ NOT LBRAC guard RBRAC\\
    subguard    &$\to$ & guard AND guard $\mid$ guard OR guard\\
    relation    &$\to$ & expr LT expr $\mid$ expr LEQ expr $\mid$ expr EQ expr $\mid$ expr NEQ expr $\mid$ expr GEQ expr\\
                &      & $\mid$ expr GT expr
  \end{tabular}
  \end{center}

Implement \texttt{generator.JasminGenerator.translateProgram(ASTNode)} which given the \emph{program} node in an abstract syntax tree returns the generated Jasmin Code as a string.

\emph{Hint: It is a good approach to write methods for every language construct and call them recursively (similar to a recursive descent parser). Once you get the idea, it is actually less effort than you might think! And do not forget to zero-initialize newly declared variables!}

Methods for generating \texttt{Jasmin} code for the main class, for writing to and reading from the console are already provided. You should also implement the method \texttt{translateExpr(ASTNode)} which translates an \emph{expression} into \texttt{Jasmin} code and is used in the method for writing to the console.

Here are some helpful resources for the \texttt{Jasmin} language:
  \begin{itemize}
    \item Jasmin main page: \href{http://jasmin.sourceforge.net/}{http://jasmin.sourceforge.net/}
    \item Jasmin user guide: \href{http://jasmin.sourceforge.net/guide.html}{http://jasmin.sourceforge.net/guide.html}
    \item List of machine instructions for Jasmin: \href{http://jasmin.sourceforge.net/instructions.html}{http://jasmin.sourceforge.net/instructions.html}
    \item Explanation of the machine instructions: \href{https://docs.oracle.com/javase/specs/jvms/se8/html/jvms-6.html}{https://docs.oracle.com/javase/specs/jvms/se8/html/jvms-6.html}
  \end{itemize}

  To test your implementation you can write code in the \textit{WHILE} language and run it through our compiler with:
  \begin{lstlisting}[basicstyle=\ttfamily,numbers=none]
$java -cp bin Main tests/gcd.txt --out gcd.j
   \end{lstlisting}
  The generated code is written to the given filename (in this example \texttt{gcd.j}). Next you can use \texttt{Jasmin} to build an executable Java class file:
  \begin{lstlisting}[basicstyle=\ttfamily,numbers=none]
$java -jar lib/jasmin.jar gcd.j
   \end{lstlisting}
You can execute the generated class file as a Java program and observe its behavior:
  \begin{lstlisting}[basicstyle=\ttfamily,numbers=none]
$java gcd
42
27
GCD: 
3
   \end{lstlisting}

  After finishing this exercise you now have your own simple compiler for Java code!
\end{exercise}

\begin{solution}
  For the solution see the code in \texttt{i2CompilerSolution}.
\end{solution}

\end{document}
