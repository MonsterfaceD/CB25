%
%
\begin{exercise}{40}
  Reconsider the attribute grammar $G$ for type checking from Example 11.5. Now consider the following program $c$:
  %
  \begin{align*}
     &\textsf{int}~y;~\textsf{bool}~x; \\
     %
     &y:=1; \\
     %
     &x:=5+y<7
  \end{align*}
  with the following syntax tree, where we enumerated all nodes:
  %
  \begin{center}
  \begin{tikzpicture}
    [scale=0.8, every node/.style={transform shape}, node distance = 1.5cm, auto, ->,>=stealth',shorten >=1pt,auto,semithick]
    %    \tikzstyle{every state}=[circle, radius = 2pt]
    \tikzstyle{state}=[circle,draw=black,fill=gray,minimum size=5pt,inner sep=0pt]
    
    \node[] (0) {Pgm, 0};
    
        \node[] (1) [below left of=0] {Dcl, 1};
    
            \node[] (2) [below left of=1] {Typ, 2};
                \node[] (3) [below of=2] {int, 3};
            \node[] (4) [right of=2] {var, 4};
            \node[] (5) [right of=4] {;, 5};
            \node[] (6) [right of=5] {Dcl, 6};
            
                \node[] (7) [below left of=6] {Typ, 7};
                    \node[] (8) [below of=7] {bool, 8};
                \node[] (9) [right of=7] {var, 9};
                \node[] (10) [right of=9] {;, 10};
                \node[] (11) [right of=10] {Dcl, 11};
                
                    \node[] (12) [below of=11] {$\varepsilon$, 12};       
  
  
        \node[] (13) [above right of=6] {Cmd, 13};
        
            \node[] (14) [right of=6] {var, 14};
            \node[] (15) [right of=14] {:=, 15};
            \node[] (16) [right of=15] {Exp, 16};
                \node[] (17) [below of=16] {num, 17};
            
            \node[] (18) [right of=16] {;, 18};
            \node[] (19) [right of=18] {Cmd, 19};
            
                \node[] (20) [below left of=19] {var, 20};
                \node[] (21) [right of=20] {:=, 21};
                \node[] (22) [right of=21] {Exp, 22};
                    \node[] (23) [below of=20] {Exp, 23};
                        \node[] (24) [below left of=23] {Exp, 24};
                            \node[] (25) [below of=24] {num, 25};
                        \node[] (26) [right of=24] {$+$, 26};
                        \node[] (27) [right of=26] {Exp, 27};
                            \node[] (28) [below of=27] {var, 28};
                    \node[] (29) [right of=23] {$<$, 29};
                    \node[] (30) [right of=29] {Exp, 30};
                        \node[] (31) [below of=30] {num, 31};
                        
                \node[] (32) [right of=22] {;, 32};
                \node[] (33) [right of=32] {Cmd, 33};
                    \node[] (34) [below of=33] {$\varepsilon$, 34};
                    
                    
    \path
    (0)  edge (1)
         edge (13)
    (1)  edge (2)
         edge (4)
         edge (5)
         edge (6)
    (2)  edge (3)
    (6)  edge (7)
         edge (9)
         edge (10)
         edge (11)
    (7)  edge (8)
    (11) edge (12)
    (13) edge (14)
         edge (15)
         edge (16)
         edge (18)
         edge (19)
    (16) edge (17)
    (19) edge (20)
         edge (21)
         edge (22)
         edge (32)
         edge (33)
    (22) edge (23)
         edge (29)
         edge (30)
    (23) edge (24)
         edge (26)
         edge (27)
    (24) edge (25)
    (27) edge (28)
    (30) edge (31)
    (33) edge (34)
    ;
  \end{tikzpicture}
  \end{center}

  \begin{enumerate}[(a)]
    %
    %
    \item Perform attribute evaluation by topological sorting. For that, provide a table of the form \\
    %
    \begin{center}
    \begin{tabular}{c | c }
    	 Node of $D_t$ & Evaluation \\
    	 \hline\hline \\
    	 %
    	 \vdots & \vdots
    \end{tabular}
    \end{center}
    %
    such that reading the entries of the first column from top to bottom yields a topological sorting of $D_t$ and where the second column gives the corresponding attribute evaluations.
    %
    \item Perform attribute evaluation by recursive functions, starting with the function call \textlangle ok, Pgm\textrangle(0). Provide a table similar to task \textbf{(a)}, but now provide the function call and its return value, i.e. of the form
    %
    \begin{center}
    \begin{tabular}{c | c }
    	 Function Call & Return Value \\
    	 \hline\hline \\
    	 %
    	 \textlangle ok, Pgm\textrangle(0) & \dots \\
    	 \vdots & \vdots
    \end{tabular}
    \end{center}
    %
    such that the table is ordered by the iterative order every functions is called in a valid recursive evaluation. You may use indents to format your function calls.
    %
    \item Is $G$ an $L$-attributed grammar? Justify your answer.
    %
    \item Is it possible to use LL parsing with integrated attribute evaluation (cf.\ Slide 13.25) for this grammar? Justify your answer.
    %
  \end{enumerate}
\end{exercise}
%
%
%
\begin{solution}
  %
  \begin{enumerate}[(a)]
    \item 
    %
    \begin{center}
    	\begin{tabular}{c | c }
    		Node of $D_t$ & Evaluation \\
    		\hline\hline \\
    		%
    		typ.2 & int \\
    		%
    		typ.7 & bool \\
    		%
    		st.11 & $\{x \mapsto err, y \mapsto err\}$ \\
    		%
    		st.6 & $\{x \mapsto bool, y \mapsto err\}$ \\
    		%
    		st.1 & $\{x \mapsto bool, y \mapsto int\}$ \\
    		%
    		env.13 & $\{x \mapsto bool, y \mapsto int\}$ \\
    		%
    		env.16 & $\{x \mapsto bool, y \mapsto int\}$ \\
    		%
    		typ.16 & int \\
    		%
    		env.19 & $\{x \mapsto bool, y \mapsto int\}$ \\
    		%
    		env.22 & $\{x \mapsto bool, y \mapsto int\}$ \\
    		%
    		env.23 & $\{x \mapsto bool, y \mapsto int\}$ \\
    		%
    		env.24 & $\{x \mapsto bool, y \mapsto int\}$ \\
    		%
    		typ.24 & int \\
    		%
    		env.27 & $\{x \mapsto bool, y \mapsto int\}$ \\
    		%
    		typ.27 & int \\
    		%
    		typ.23 & int \\
    		%
    		env.30 & $\{x \mapsto bool, y \mapsto int\}$ \\
    		%
    		typ.30 & int \\
    		%
    		env.33 & $\{x \mapsto bool, y \mapsto int\}$ \\
    		%
    		ok.33 & true \\
    		%
    		ok.19 & true \\
    		%
    		ok.13 & true \\
    		%
    	     ok.0 & true
    	\end{tabular}
    \end{center}
    %
    \item 
        Let $\alpha=\{x \mapsto bool, y \mapsto int\}$.
        \begin{center}
        \begin{tabular}{l | c }
    	    Function Call & Return Value \\
    	    \hline\hline \\
    	    %
    	    \textlangle ok, Pgm\textrangle(0) & true \\
    	    \quad\textlangle st, Dcl\textrangle(1) & $\{x \mapsto bool, y \mapsto int\}=\alpha$\\
    	    \quad\quad\textlangle st, Dcl\textrangle(6) & $\{x \mapsto bool, y \mapsto err\}$\\
    	    \quad\quad\quad\textlangle st, Dcl\textrangle(11) & $\{x \mapsto err, y \mapsto err\}$\\
    	    \quad\quad\quad\textlangle typ, Typ\textrangle(7) & bool\\
    	    \quad\quad\textlangle typ, Typ\textrangle(2) & int\\
    	    \quad\textlangle ok, Cmd\textrangle(13, $\alpha$) & true \\
    	    \quad\quad\textlangle typ, Exp\textrangle(16, $\alpha$) & int\\
    	    \quad\quad\textlangle ok, Cmd\textrangle(19, $\alpha$) & true \\
    	    \quad\quad\quad\textlangle typ, Exp\textrangle(22, $\alpha$) & bool \\
    	    \quad\quad\quad\quad\textlangle typ, Exp\textrangle(23, $\alpha$) & int \\
    	    \quad\quad\quad\quad\quad\textlangle typ, Exp\textrangle(24, $\alpha$) & int \\
    	    \quad\quad\quad\quad\quad\textlangle typ, Exp\textrangle(27, $\alpha$) & int \\
    	    \quad\quad\quad\quad\textlangle typ, Exp\textrangle(30, $\alpha$) & int \\
    	    \quad\quad\quad\textlangle ok, Cmd\textrangle(33, $\alpha$) & true \\
        \end{tabular}
        \end{center}
    %
    \item Yes: Whenever we consider an equation with an inherited attribute on the left-hand side, then the synthesized attributes occurring on the right-hand side has a strictly smaller index: The only such equation is $env.2 = st.1$, however this satisfies the condition.
    %
    \item Since we have left-recursion in the Exp part of the grammar, we have $G\not \in LL(k)$ for all $k\in \Nats$. It is thus not possible to use LL parsing for this grammar.
    %
    %
  \end{enumerate}
\end{solution}
