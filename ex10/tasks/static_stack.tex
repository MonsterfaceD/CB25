\begin{exercise}{40}
  In this task, we extend our intermediate language EPL by another \emph{static} stack data structure
  (in addition to those given in Definition~17.4). 
  A static stack $S$ has a fixed (i.e., determined at compile time) puffer length $z$ and stores elements of type $I$. 
  It is declared as follows:
  %
  %
  \begin{center}
    \texttt{stack(z) of $I$}
  \end{center}
  Furthermore, a static Stack $S$ provides the four methods
  \begin{center}
  \begin{tabular}{ll}
    push:    & \texttt{$S+$}\\
    pop:     & \texttt{$S-$}\\
    isEmpty: & \texttt{$S?-$}\\
    isFull:  & \texttt{$S?+$}~,
  \end{tabular}
  \end{center}
  where $S+$ must occur on the left-hand side on an assignment only. $S-$, $S?-$ and $S?+$ must occur on the right-hand side on an assignment and in boolean expressions for if-statements and while-statements only. We assume for simplicity that if $S+$ is on the left hand side of the assignment, then $S-$ can not occur on the right hand side of this assignment.
  \begin{enumerate}[(a)]
    \item Extend the symbol table functions $update$ and $vtype$ accordingly.
    \item Extend the translation for variables function $vt$ for $S+$ and $S-$ and $et$ for $S?+$ and $S?-$ accordingly.
    \item Generate the AM code for the following EPL program:
      \begin{center}
        \texttt{type IntStack = stack(5) of int; var t: int, S: IntStack; S+ := 1; t := S- + 1;}
      \end{center}
      That is, compute $\mathrm{trans}(...)$ of the above program.
      %
      \item Describe how we can realize \emph{jumping code} for Boolean expressions in our extended EPL. In particular, provide the $sbt$ functions for $S?-$ and  $S?+$.
  \end{enumerate}
\end{exercise}

\begin{solution}
  \begin{enumerate}[(a)]
    \item
    
    update(type I := \texttt{stack(z) of $J$}; $D_T'$, $st$) := \\
    \hspace*{5em} update(type $D_T'$, $st[I \mapsto (type,stack,J,z,(z\cdot m)+1)]$)\\
    if $st(J)=(type, \dots, m)$\\
    
    \begin{tabular}{lcl}
        vtype($S-$, st)  & := & $J$ \\
                         &    & if $vtype($S$,st)=I, st(I)=(type,stack,J,z,n)$\\
        vtype($S+$, st)  & := & $J$ \\
                         &    & if $vtype($S$,st)=I, st(I)=(type,stack,J,z,n)$\\
        vtype($S?-$, st) & := & bool \\
        vtype($S?+$, st) & := & bool \\
    \end{tabular}
    
    \item We assume for this solution that at the start of the program the memory locations are initialized with $0$, i.e. $\sigma(i)=0$.
    We first define a helper function $ls$ that loads the non-relative stack pointer address:\\
    \begin{tabular}{lcll}
        ls($S$, st, a) & := & vt($S$, st, a)  & \% base address \\
                       &    & vt($S$, st, a') & \% pointer address \\
                       &    & LOAD;           & \% load pointer \\
                       &    & CAB(1,z)        & \% check pointer range \\
                       &    & LIT(m); MULT;   & \% relative address \\
                       &    & ADD;            & \% address for $S+$ \\
    \end{tabular}\\
    if $vtype($S$,st) = I, st(I) = (type,stack,J,z,n), st(J) = (type, \dots, m)$
    
    Now we use $ls$ to define $vt$:\\
    \begin{tabular}{lcll}
        vt($S+$, st, a) & := & vt($S$, st, a)   & \% pointer address \\
                        &    & vt($S$, st, a')  & \% pointer address \\
                        &    & LOAD;            & \% load pointer \\
                        &    & LIT(1); ADD;     & \% increment pointer \\
                        &    & STORE;           & \% store back again \\
                        &    & ls($S$, st, a'') & \% compute stack pointer address \\
    \end{tabular}\\
    
    \begin{tabular}{lcll}
        vt($S-$, st, a) & := & ls($S$, st, a)   & \% compute stack pointer address \\
                        &    & vt($S$, st, a')  & \% pointer address \\
                        &    & vt($S$, st, a'') & \% pointer address \\
                        &    & LOAD;            & \% load pointer \\
                        &    & LIT(1); SUB;     & \% decrement pointer \\
                        &    & STORE;           & \% store back again \\
    \end{tabular}\\
    
    \begin{tabular}{lcll}
        et($S?+$, st, a) & := & vt($S$, st, a) & \% pointer address \\
                         &    & LOAD;          & \% load pointer \\
                         &    & LIT(z);        & \% load max size \\
                         &    & GEQ;           & \% is pointer greater or equal than z? \\
    \end{tabular}\\
    if $vtype($S$,st) = I, st(I) = (type,stack,J,z,n)$\\
    
    \begin{tabular}{lcll}
        et($S?-$, st, a) & := & a: LIT(0);      & \% load min size \\
                         &    & vt($S$, st, a') & \% pointer address \\
                         &    & LOAD;           & \% load pointer \\
                         &    & GEQ;            & \% is 0 greater or equal than pointer? \\
    \end{tabular}




    \item The computed program is:\\

      \begin{tabular}{rcll}
         1 & : \hspace{0.3cm} & LIT (1); &\% increment $S$\\
         2 & : \hspace{0.3cm} & LIT (1); &\\
         3 & : \hspace{0.3cm} & LOAD; &\\
         4 & : \hspace{0.3cm} & LIT (1); &\\
         5 & : \hspace{0.3cm} & ADD; &\\
         6 & : \hspace{0.3cm} & STORE; &\\
         
         7 & : \hspace{0.3cm} & LIT (1); &\% compute stack address\\
         8 & : \hspace{0.3cm} & LIT (1); &\\
         9 & : \hspace{0.3cm} & LOAD; &\\
         10 & : \hspace{0.3cm} & CAB(1,5); &\\
         11 & : \hspace{0.3cm} & LIT (1); &\\
         12 & : \hspace{0.3cm} & MULT; &\\
         13 & : \hspace{0.3cm} & ADD; &\\
         
         14 & : \hspace{0.3cm} & LIT (1); &\% store 1 on $S+$\\
         15 & : \hspace{0.3cm} & STORE; &\\
         
         16 & : \hspace{0.3cm} & LIT (0); &\% access t \\ 
         
         17 & : \hspace{0.3cm} & LIT (1); &\% compute stack address\\
         18 & : \hspace{0.3cm} & LIT (1); &\\
         19 & : \hspace{0.3cm} & LOAD; &\\
         20 & : \hspace{0.3cm} & CAB (1,5); &\\
         21 & : \hspace{0.3cm} & LIT (1); &\\
         22 & : \hspace{0.3cm} & MULT; &\\
         23 & : \hspace{0.3cm} & ADD; &\\
        
         24 & : \hspace{0.3cm} & LIT (1); &\% decrement $S$\\
         25 & : \hspace{0.3cm} & LIT (1); &\\
         26 & : \hspace{0.3cm} & LOAD; &\\
         27 & : \hspace{0.3cm} & LIT (1); &\\
         28 & : \hspace{0.3cm} & SUB; &\\
         29 & : \hspace{0.3cm} & STORE; &\\
         
         30 & : \hspace{0.3cm} & LOAD; &\% load stack\\
         
         31 & : \hspace{0.3cm} & LIT (1); &\% add 1\\
         32 & : \hspace{0.3cm} & ADD; &\\
         
         33 & : \hspace{0.3cm} & STORE; &\% store in t\\
         34 & : \hspace{0.3cm} & RET; &\% end\\
      \end{tabular}
  
     \item We define $sbt(S?-, st, a, a_t, a_f)$ and $sbt(S?-, st, a, a_t, a_f)$ by
     
     %
     \begin{tabular}{lcll}
     	sbt($S?+, st, a, a_t, a_f$) & := & vt($S$, st, a)   & \% pointer address, pc starts at $a$ \\
     	&    & LOAD;         & \% load pointer \\
     	&    & LIT(z);       & \% load max size \\
     	&    & GEQ;          & \% is pointer greater or equal than z? \\
     	&    &JFALSE($a_f$); & \% Jump to address for the false case \\
     	&    & JMP($a_t$); & \% Otherwise, jump to address for true case
     \end{tabular}\\
     if $vtype($S$,st)=I, st(I)=(type,stack,J,z,n)$\\
     
     \begin{tabular}{lcll}
     	sbt($S?-, st, a, a_t, a_f$) & := & a : LIT(0);       & \% load min size, pc starts at $a$ \\
     	&    & vt($S$, st, a+1)   & \% pointer address, pc starts at $a+1$ \\
     	&    & LOAD;         & \% load pointer \\
     	&    & GEQ;          & \% is 0 greater or equal than pointer? \\
     	&    &JFALSE($a_f$); & \% Jump to address for the false case \\
     	&    & JMP($a_t$); & \% Otherwise, jump to address for true case
     \end{tabular}
    
    Moreover, $sbt$ for the composite Boolean expressions is defined analogous to Definition 17.1 (we omit the last component for the level $l$ since we do not consider procedures here.)
 
     
     
  \end{enumerate}
\end{solution}
